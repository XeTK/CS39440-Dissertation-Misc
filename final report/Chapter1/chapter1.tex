\chapter{Background \& Objectives}

The primary goal of this project was to create a fun and interesting manor to send messages between friends, while trying to design and implement something new and ingenuity.\\
\\
 Location Sensitive Social Notifier or Lo Se Sono for short offers the opportunity to be a completely new platform to communicate messages between people and be a potential public service that allows messages to be shared within a local area.

\section{Background}

The idea behind the project was to create a location based messaging application where users could leave tags on specific locations on a map and it be picked up by the users friends when they pass over the location, this sounds fairly mundane but it was a idea that had not been done without the help of external devices like blue tooth tags.\\
\\
Conceptually the application is interesting as it allows people to place a message for example outside of a shop placing a tag saying that a product within the shop is heavily discounted or as an another example if a person was participating within a marathon and the organisers could add tags to specific points throughout the route to indicate how far a participant has completed, theses messages could include useful information indicating that there is water near by or other interesting facts.\\
\\
More interestingly friends groups could use it within there groups to indicate where users are meeting and sharing useful facts among there friend groups, overall this is a fun concept that could be used for multiple different situations and we are unlikely to know all of the applications uses till it enters a production like environment where there are real users trying new things.\\
\\
The concept takes hints from other applications and products to create a somewhat social media type style to its implementation taking features from Facebook, Snapchat and Imgur just to name a few, with strong UI cues taken from Snapchat, with the commenting and voting system closely inspired by Imgur.

\subsection{Origin}

The origin of Lo Se Sono stems back to a sleepless night where the idea for an GPS based application that would buzz in users pocket when they were walking close to there friends to notify them to the fact that there was one of there friends near by, this flowed into what if a user could leave tags/messages in a given location and there friends were notified of this tag/message that had been left, thinking this would be a very useful idea for a group of friend to remind another member to collect a specific item.\\
\\
The idea was left for 6-7 months before thoughts of dissertations resurfaced and it seemed this would be a well rounded and interesting project to implement. There were slight changes made to the concept throughout the early stages as it was initially suggested to develop the idea with the use of 3rd party blue-tooth dongles to give a much more accurate fix on location, but it would mean that not everyone with a smart phone could use the application the idea of having extra hardware would limit the applications audience and make it difficult for the application to gain critical motion if the application needs extra hardware to work but the GPS does mean the accuracy of the application is hurt and will only be accurate to at most 15 meters away from the object in any direction.

\subsection{Research}

Researching the project consisted of researching if the idea had been implemented before, there were a few similar ideas but nothing that did it well or had major design differences one of the similar products was named Message Drop \cite{dmt:dropmessageteam:2015:online} and was a Apple iOS exclusive. There were a few other projects that bad been around in the past but seemed to have died off or evolved into other projects.\\
\\
Once the market research was completed attention turned to the different technologies that would have to underpin the application, most specifically GPS and which mobile operating system to code the application for the two main contenders were Android and Phonegapp in depth coverage of each platform can be found in Chapter \ref{ch:design} section \ref{sec:android_choice_of_tech}. Researching GPS was key to the applications success and factors that needed to assessed included its overall accuracy which some indicate can be between 2-40 meters \cite{DevdattaTengshe:gpsacuracy:2012:online} depending on conditions it became clear throughout the project that this is a much bigger problem than first through and it is clear the GPS struggles to get a fix in anything but near perfect environments.\\
\\
Other extensive research was taken into looking into the best ways to design Google Android applications with a fair bit of studying of Google's own material design guides to ensure the application had a uniform look and would fit in with the Android ecosystem.\\
\\
There was considerable time taken to deciding what languages \& frameworks to use for the project weighing up the correct framework to provide the RESTful interfaces for the application in depth explanation of the rational behind each choice in framework can be found in Chapter \ref{ch:design}.

\section{Analysis}

{TODO}

\subsection{Functional Requirements}

{TODO}

\subsection{Technical Requirements}

{TODO}


\section{Process}

{TODO}

\subsection{Agile}

{TODO}

\subsection{Feature Driven Development}

{TODO}

\subsection{eXtream Programming}

{TODO}