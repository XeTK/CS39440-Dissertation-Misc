\chapter{Evaluation}

This chapter will give critical evaluation of the project as a whole along with giving feedback on the various elements that made up the project.

\section{Application}

First of critical evaluation will take place of the application that was developed as part of the project, illustrating the good points and the bad points of the application explaining what was done well within the application and what could do with improvement to make the application better as a whole.

\subsection{Achievements - Good Bits}

This section will talk about bits of the application that were through to have been executed well or were positive parts of the project. It will also discus the parts of the project that the developer was specifically proud of.

\subsubsection{User Feedback}

The overall feedback that was received from the users was very good and gave a good indication that the application was on the right route to becoming and application that users want to use for the various location based messages, Thomas Nicolaides wrote "I think the appliation is a great concept, and it fills a niche that nothing I've found previously has. I definitely would use it when it becomes available to more people." which gives a great idea that the concept is something that users want to use Daniel Jones wrote "{STICK QUOTE IN}" .\\
\\
This feedback is a great accomplishment and it is great to see the user community is interested in the application as a whole and that it could potentially grow into a application that a lot of people want to use. A large part of general feedback received from showing the application has been fairly positive and people seemed to be very intrigued by the idea and want to see a fully fledged application to try out.

\subsubsection{Database Design}

The developer of the application is very happy with the way the database has been designed and the design created at the very start of the project seems to have exceeded its expectations with very minimal modifications being made to it to make it suit application design as the design has matured and grown into the fully fledged application that we see detailed in this document.\\
\\
The design is very robust and copes well with the growth of the application and it should be very easy to add on new functionality to the database as the application grows and matures to become a production ready application, its simplistically and low coupling should help with other developers understanding how the everything links together.\\
\\
Database functions also help to complete the database part of the application, helping keeping database type tasks segregated from the middle tier application this should intern help with security and performance which are key points to the application that want to be kept robust as possible to help with the stress of users using the application.

\subsubsection{Choice Of Technologies}

For the most part the choice of technology was right, the use of PostgreSQL and HAPI.js/Node.js suited the backend development down to the ground making development fast along with providing the features that were needed to get the job done quickly and effectively.\\
\\
PostgreSQL lived completely up to the developers expectations and gave many more useful features than expected its robustness can not be faulted for this project along with the advanced features offered included integrated JSON parser, advanced database functions along with built in security features with advanced hashing and encryption algorithms which made making the data secure a breeze.\\
\\
HAPI.js and Node.js together made the perfect backend for the application, being extremely flexible and versatile with a great developer community behind them which were raring to help anyone using there API's. Theses frameworks worked perfectly and did everything that was asked of them in well structured and well documented ways which ultimately helped speed up the development of the application and create a better product at the end of it. The stability of the Node.js platform has been excellent as well with very minor issues that stopped the application behaving as intended.

\subsubsection{Dynamic Loading}

Dynamic loading of routes within the backend application was a great time saver and a really interesting concept to get working, the developer may not have developed it exclusively for this project but it was suited perfectly for loading in the routes for the RESTful services, which meant that no modifications were needed to be made to the supporting code to make it load in the new routes into the application which add the additional functionality needed to carry out the advanced features within the GUI.\\
\\
The use of a Dynamic Route Loader should help make the application more flexible to change in future and allow people to add on new functionality with ease without having to modify the underlying code, this should also mean that it will be very difficult for new bits of code to add in bugs to the underlying infrastructure.

\subsubsection{Viewing \& Posting Tags}
\label{sec:viewing_posting}

One of the best bits of the application is the ability to quickly and easily post new messages within the application with a very clear and simple UI that shows exactly where the post will be placed on the map with a simplistic text entry system that allows simple and very straight forward ways to add in a new message to the map.\\
\\
The UI for viewing a messages is as simple as it can be and shows all the relevant information needed to portray the information assigned to the location on the map, with the ability to leave comments and vote on the specific point being very interesting or not and this is a great way to get the users to interact with the various points that have been left on the map which will be important to get users to interact with the application and ultimately build the customer base.\\
\\
The map that shows exactly where the tag that has been left on the map and is a very good visual cue on the viewing messages screen to illustrate where the tag is and what it means to the user. This helps break up the screen and helps to keep the idea interesting even if the user is not interested in the social and commenting side of the application.

\subsubsection{Comments \& Voting}

Commenting and voting as mentioned in the section \ref{sec:viewing_posting} are very integral to the users participation within the application and are some of the features that the developer is most proud of as they give a framework for users to collaborate and to leave insightful comments on the posts that have been left on the map.\\
\\
Getting the user experience right for comments was fairly complicated procedure as there are a lot of UI elements that make up the comment sections within the application and getting them all to align and work together is a fairly difficult procedure as there also many different screen resolutions to tailor the application to work with.

\subsubsection{Notifications}

The notification section of the application is a fairly well throughout and interesting part of the application it uses some interesting algorithms to ensure it notify the users at the correct points in time, it does need some more work to make it a fully rounded part of the application but it is clear to see the potential of the notifications within the application, they give clear alerts that tell the user exactly what is needed to be known to tell them about near by tags.\\
\\
Notifications do need some work to make them better at saving power and clearing at the appropriate times, but it is clear to see how they engage the users to interact with the application and actually get a response out of the user. With a bit more work it will be a integral part of the application that users will appreciate.

\subsubsection{Project Concept}

The greatest success of the project has to defiantly the concept behind the application, it seemed to not been done prior to implementing this project and the proof of concept application created from this project seems to prove the idea works and that people want to use the application.\\
\\
With some further development it is clear that the concept can be turned into a product that people will want to use and could quite potentially become a fairly popular application out in the market place which would be a crowning achievement of the project and would make the developer immensely proud.\\
\\
The concept has come a long way since the early development of the project and it will be interesting to see how the application matures with time.

\subsection{Limitations \& Potential improvements - Bad bits}

In this section the limitations of the application and what could be improved will be discussed, it will detail what parts of the application were not up to the standard of the developer.

\subsubsection{Testing}

\subsubsection{Android}

% Talk about phone gapp being the same language and framework as backend...

\subsubsection{Confusing UI}

\subsubsection{Actively Negative}

\subsubsection{Error handling}

\subsubsection{Mapping With Multiple Pins}

\subsubsection{Application Completeness}

\subsubsection{GPS accuracy}


\section{Developers performance}

This section will be assessing the developers performance throughout the project, giving critical feedback about what was done well and what could have been done better to make the project better as a whole.

\subsection{Positive - what was done well}

This section will cover in detail what the developer thought what they did well to help with the development of the project and what they were specifically proud of.

\subsubsection{Risk Assessment}

\subsubsection{Priority Assessment}

\subsubsection{Test Community}

\subsubsection{Problem Solving}


\subsection{Negative - what could be improved}

Contained within this section is what the developer thought they could of possibly done better to aid with the development of the project as a whole and ultimately created a better product. 

\subsubsection{Overly Critical}

\subsubsection{Time Management}

\subsubsection{Methodologies}

\subsubsection{Motivation}


\section{Overview}

