\chapter{Evaluation}

This chapter will give critical evaluation of the project as a whole along with giving feedback on the various elements that made up the project.

\section{Application}

Firstly, critical evaluation will take place of the application that was developed as part of the project, illustrating the good points and the bad points of the application, explaining what was done well within the application and what could do with improvement to make the application better as a whole.

\subsection{Achievements - Good Bits}

This section will talk about parts of the application that were thought to have been executed well or were positive parts of the project. It will also discus the parts of the project that the developer was specifically proud of.

\subsubsection{User Feedback}

The overall feedback that was received from the users was very good and gave a good indication that the application was on the right route to becoming an application that users want to use for the various location based messages. Thomas Nicolaides wrote "I think the application is a great concept, and it fills a niche that nothing I've found previously has. I definitely would use it when it becomes available to more people." This gives a good idea that the concept is something that users want to use. Daniel Jones wrote "{STICK QUOTE IN}" .\\
\\
This feedback is a great accomplishment and it is good to see the user community is interested in the application as a whole and that it could potentially grow into an application that a lot of people want to use. A large part of the general feedback received from showing the application has been fairly positive and people seemed to be very intrigued by the idea and want to see a fully fledged application to try out.

\subsubsection{Database Design}

The developer of the application is very happy with the way the database has been designed and the design created at the very start of the project seems to have exceeded its expectations, with very minimal modifications being made to it to make it suit application design. The design has matured and grown into the fully fledged application that we see detailed in this document.\\
\\
The design is very robust and copes well with the growth of the application and it should be very easy to add on new functionality to the database as the application grows and matures to become a production ready application. Its simplicity and low coupling should help with other developers' understanding how everything links together.\\
\\
Database functions also help to complete the database part of the application, helping to keep database type tasks segregated from the middle tier application. This should in turn help with security and performance which are key points to the application that want to be kept as robust as possible to help with the stress of users using the application.

\subsubsection{Choice Of Technologies}

For the most part the choice of technology was right. The use of PostgreSQL and HAPI.js/Node.js suited the backend development down to the ground, making development fast, along with providing the features that were needed to get the job done quickly and effectively.\\
\\
PostgreSQL lived completely up to the developer's expectations and gave many more useful features than expected. Its robustness can not be faulted for this project, along with the advanced features offered that included integrated JSON parser, advanced database functions along with built in security features with advanced hashing and encryption algorithms which made making the data secure a breeze.\\
\\
HAPI.js and Node.js together made the perfect backend for the application, being extremely flexible and versatile, with a great developer community behind them which were rearing to help anyone using their API's. These frameworks worked perfectly and did everything that was asked of them in well structured and well documented ways, which ultimately helped speed up the development of the application and create a better product at the end of it. The stability of the Node.js platform has been excellent as well, with very minor issues that stopped the application behaving as intended.

\subsubsection{Dynamic Loading}

Dynamic loading of routes within the backend application was a great time saver and a really interesting concept to get working. The developer may not have developed it exclusively for this project, but it was suited perfectly for loading in the routes for the RESTful services, which meant that no modifications were needed to be made to the supporting code to make it load in the new routes into the application which add the additional functionality needed to carry out the advanced features within the GUI.\\
\\
The use of a Dynamic Route Loader should help make the application more flexible to change in future and allow people to add on new functionality with ease, without having to modify the underlying code. This should also mean that it will be very difficult for new bits of code to add in bugs to the underlying infrastructure.

\subsubsection{Viewing \& Posting Tags}
\label{sec:viewing_posting}

One of the best bits of the application is the ability to quickly and easily post new messages within the application with a very clear and simple UI that shows exactly where the post will be placed on the map, with a simplistic text entry system that allows simple and very straightforward ways to add in a new message to the map.\\
\\
The UI for viewing messages is as simple as it can be and shows all the relevant information needed to portray the information assigned to the location on the map, with the ability to leave comments and vote on the specific point being very interesting or not. This is a great way to get the users to interact with the various points that have been left on the map, which will be important to get users to interact with the application and ultimately build the customer base.\\
\\
The map shows exactly where the tag has been left on the map and is a very good visual cue on the viewing messages screen to illustrate where the tag is and what it means to the user. This helps break up the screen and helps to keep the idea interesting, even if the user is not interested in the social and commenting side of the application.

\subsubsection{Comments \& Voting}

Commenting and voting as mentioned in the section \ref{sec:viewing_posting} are very integral to the user's participation within the application and are some of the features that the developer is most proud of, as they give a framework for users to collaborate and to leave insightful comments on the posts that have been left on the map.\\
\\
Getting the user experience right for comments was a fairly complicated procedure as there are a lot of UI elements that make up the comment sections within the application and getting them all to align and work together is a fairly difficult procedure as there also many different screen resolutions to tailor the application to work with.

\subsubsection{Notifications}

The notification section of the application is a fairly well thought out and interesting part of the application. It uses some interesting algorithms to ensure it notifies the users at the correct points in time. It does need some more work to make it a fully rounded part of the application, but it is clear to see the potential of the notifications within the application. They give clear alerts that tell the user exactly what is needed to be known to tell them about nearby tags.\\
\\
Notifications do need some work to make them better at saving power and clearing at the appropriate times, but it is clear to see how they engage the users to interact with the application and actually get a response out of the user. With a bit more work it will be an integral part of the application that users will appreciate.

\subsubsection{Project Concept}

The greatest success of the project has to definitely be down to the concept behind the application. It seemed to not have been done prior to implementing this project and the proof of concept application created from this project seems to prove the idea works and that people want to use the application.\\
\\
With some further development it is clear that the concept can be turned into a product that people will want to use and could quite potentially become a fairly popular application out in the market place, which would be a crowning achievement of the project and would make the developer immensely proud.\\
\\
The concept has come a long way since the early development of the project and it will be interesting to see how the application matures with time.

\subsection{Limitations \& Potential improvements - Bad bits}

In this section the limitations of the application and what could be improved will be discussed, it will detail what parts of the application were not up to the standard of the developer.

\subsubsection{Testing}

Testing is one part of the project that the developer does think that needs some improvement, more focus should have been given to it in the early part of the project to ensure that the application had good test coverage.\\
\\
Automated testing would help with ensuring the robustness of the application and check if any regression had happened within the development of the application which would result in a less robust application. With more time the use of automated UI testing could of aided to see if the UI worked correctly on different devices, along with unit tests to ensure that the code base remains robust and stable.\\
\\
The testing issues can be partly blamed on the limited time given for the project and the constraints this placed on developing the application it was felt although maybe this shouldnt of been the case to focus on just developing a working application to prove the concept that trying to make the application more robust from the start with slower progress overall but leading to a much more robust platform with less functionality by this stage in the project.

\subsubsection{Android}

The initial choice to use native Google Android may been a poor judgment to have made, it required re-learning a programming language and to learn a whole new framework along side that, a significant amount of time and effort was spent trying to get a grips on Android development and with hindsight it is clear that PhoneGapp may have been a much better route to go down even if it isn't as mature as the native libraries.\\
\\
Androids innate instability and development complexities that it present are a very big turn of as a developer as a lot of time is wasted in trying to get the application just to run, and what seem to be trivial elements of the design seem to take a very long time to implement for example to implement a list within the UI requires 3 classes to make it function correctly, all with modifications to make them suite the specific need.\\
\\
Phone Gapp uses the same language and framework as the backend application Node.js and uses standard web standards for example HTML, CSS \& JavaScript for its front end work. It is unclear if using Phone Gapp would of saved any pain or time in the development of the project, the developer wishes they had done more research into Phone Gapp at the beginning of the project and had done more spike work to determine what platform would be best to developer for.

\subsubsection{Confusing UI}

Some of the feedback given by the user community that tested the application has said that the User Interface can be confusing and inconsistent at other points leading to confusion which means that people struggle to understand the application and how to do tasks within it.\\
\\
The main home page seems to have caused a lot of confusion as initially its very unclear how to post a message and how to differentiate between the different users tags. This can be very disorientating for new users as they are a bit overwhelmed to start with and do not know how to interact specifically with the application.\\
\\
With more development and maybe with a developer that is better experienced at UI design or design in general looking over the application it will be easier to design and set out a design that will be less confusing for the user to use.

\subsubsection{Actively Negative}

One query that was raised by one of the users of the application that the message \& comments systems voting system seems to be overly negative and users can be actively negative towards a post rather than passively negative by choosing not to up vote a post, they are being actively negative by pressing on the down vote button.\\
\\
By using negative voting it encourages a negative environment and could show negative reenforcement which could ultimately hurt the users of the application and they may feel that they are being actively targeted and that all of there actions are bad.

\subsubsection{Error handling}

The error handling problems are in the same vain as testing as it was felt with the application should be completed to a proof of concept stage and that any extras should be left to further development, but the developer wishes they had spent more time on making the application being robust rather than rapidly developing the project to a proof of concept stage were the underlying code is not robust and will need to be gone over thoroughly before the application could be released.\\
\\
Much tighter error handling and coping strategies need to be implemented into the application before it can be released to the public as at the moment when issues happen the application is fairly likely to fall over and not give useful feedback to the user explaining what has happened for to cause the action to fail.\\
\\
With more time and further development theses issues could easily be resolved and the application should reach a maturity that means the application can be enjoyed by the general public.

\subsubsection{GPS accuracy}

As mentioned in chapter \ref{ch:background} section \ref{sec:origin} there are some issues with the accuracy of GPS, this is not really a issue but it is a noted problem with the idea as a whole as with the accuracy as it stands its not possible to do fine marking of locations, thus would be useless if the were a load of locations very close to each other that need to be marked by users differentiate between them, for example if there was a row of shops and each of them had a marker outside them telling what they are.\\
\\
It could be possible to use a Blue Tooth dongle with the application to give much more precise positioning, but this would require extra work and defeats the concept of the application working without any external hardware which was one of the main aims of the project as a whole.

\subsubsection{Mapping With Multiple Pins}

This is a issue with the Google Maps API rather than the application but there is a issue where that if there are multiple tags on the same location then its very hard to differentiate between the different tags that have been left on the same placed.\\
\\
The issue could be resolved with a bit of creative thinking from the developers end having a pin that pulls up a notification list with all the tags that are stored in the same area, this problem can be resolved in further development like most of the flaws within the application, more time is needed to polish the application.

\subsubsection{Application Completeness}

As mentioned in multiple other sections of this document the applications is far from being complete to a level where the developer is happy to release the product to the general public, the application that will be included with this report shows that the concept works and has great potential, but ultimately the application will require fair amount of work to get it to the state where it can be used by members of the public.\\
\\
Without the delays throughout the project the developer does not think that the application would be much more complete than it already is and it genuinely requires a fair amount of time to be spent to make it a viable product, the end of the dissertation does not mean the end of this project the developer has decided that they want to continue development of the application and hopefully create something that will be used a large amount of people.\\
\\
The project has had a fair bit of interest from people that are interested in helping with development and create a application and community that will work in the real world and will be a viable idea for real people.


\section{Developers performance}

This section will be assessing the developer's performance throughout the project, giving critical feedback about what was done well and what could have been done better to make the project better as a whole.

\subsection{Positive - what was done well}

This section will cover in detail what the developer thought that they did well to help with the development of the project and what they were specifically proud of.

\subsubsection{Risk Assessment}

For the project it was apparent that some risks would have to be taken to get the application to a workable level. There were stages in the early development of the application where it was felt that some of the implementation for the application was behind schedule, thus some risk assessment was needed to ensure that the project would not fall behind in the development stages.\\
\\
At the point of the midway demo it was felt that the project was falling behind and there was not enough to present. Risk assessment was used to prioritise what parts of the application needed to be completed to have a reasonable demo to show at the midway point. It was decided that the user must be able to login by this point, along with send a message to the server with the GPS coordinates of where the user was. This functionality was successfully implemented by the time the midway demo occurred, leading to a very successful demo.

\subsubsection{Priority Assessment}

With the time constraints of the project being fairly low, a plan needed to be created to ensure that the project would not be at risk of failing. It was decided that the application should be developed to a proof of concept stage which should enable user testing to occur to give an indication of whether the concept behind the application works.\\
\\
With the limited time hard decisions were made to make sure the application would reach a point where it could be tested by real users. This meant that only the most important features stayed for the development of the application. It was essential that the posting and viewing of messages were complete, along with the ability to see the list of notifications in the surrounding areas. They are not quite as robust as the developer would like but the decision was made to prioritise getting the implementation there rather than making the application completely robust to all known issues.

\subsubsection{Test Community}

Setting up a community to test the application was a novel idea for the development of the application. It got good coverage of the application and helped uncover bugs and issues that the developer would have not thought of or encountered. This also gave a much broader range of devices to test the application on and ensure that the users had their voice heard when ideas for improvements are given.\\
\\
The test community has been a fun way to get feedback on the application and to get real user's feedback on the concept, which has helped improve motivation and willingness to work on the application, as it shows people are interested in the concept and that people actually want to use the application for real world tasks.

\subsubsection{Problem Solving}

Problem solving throughout the project has been strong. With every problem there has been a solution put in place. A large proportion of the early development of the project was spent problem solving various Android devices and getting them up to a point where they could be used for development. This required large amounts of time spent learning the SDK documentation and reading a fair number of blog posts on how to fix various issues within Android and joining the points from all these resources showed strong problem solving initiative.\\
\\
The use of strong problem solving abilities were used when trying to resolve various issues throughout the development. For example, when there were some issues with the RESTful interfaces returning the wrong status, even when they were doing the correct action. The problem looked like it was originating within the UI as when mocking the interface it looked to be returning the correct information, but after digging through each layer of the application it became apparent that the problem was residing in the backend of the application, not the front end, and with a simple tweak it was possible to get the application working correctly.

\subsection{Negative - what could be improved}
\label{sec:negative_personal}

Contained within this section is what the developer thought they could of possibly done better to aid with the development of the project as a whole and ultimately created a better product. 

\subsubsection{Methodologies}

Covered in detail in chapter \ref{ch:background} section \ref{sec:methodology}, this is the process that was used for developing the project, but the developer feels that theses should have been followed more closely to ensure that the other issues detailed within this subsection \ref{sec:negative_personal} didn't happen.\\
\\
The use of strong software engineering methodologies should of helped with dealing with the difficult problems that are encountered during any software development, the developer wishes they had followed them more closely to help with the development and ultimately deliver a better pieces of software than what was delivered.

\subsubsection{Overly Critical}

At points the developer has been overly critical of a issue or problem and has dedicated to much time to resolving what in reality is a minor issue, the developer has a perfectionist nature and wants everything to work perfectly even when it is not exactly necessary.\\
\\
A prime example of this was during the early configuration of the Android environment the developer was not happy with the version of Android being used so spent a long time trying to get the device for testing up to the version that they wanted to use even though this caused major problems and delays at the start of the project.

\subsubsection{Time Management}
\label{sec:time_management}

During the development of the project there were points where the developer struggled with time management due to other commitments and factoring in the large work load of the project around them, this also ties into section \ref{sec:motivation} about motivation as they are somewhat interlinked.\\
\\
Theses issues could be resolved by better use of methodologies and coping techniques for resolving bad time management, the bad time management has lead to features being cut when they probably shouldn't have been and has ultimately put unnecessary stress on the developer when maybe in some cases if better time management was executed this would not be the case, and the developer could of performed much better due to not being stressed due to bad time management.

\subsubsection{Motivation}
\label{sec:motivation}

Motivation is a difficult thing for a simple person project, the developer finds it very difficult to develop an application without an outside customer as at the end of the day the customer is the developer. There is no outside force to push the project along it is all down to a single developer, who may have bad patch which will lead to a de-motivated streak which ultimately slows down the project and leads to the project falling behind which then leads back to the time management issues spoke about in section \ref{sec:time_management}.\\
\\
With better use of methodologies and maybe some 3rd party input the developer could of thought through these de-motivated timed and ultimately improved the productivity and kept the project on target, even while they were experiencing personal problems.

\section{Overview}

Overall the development of this project has been very rewarding and I as a developer has learnt a lot about developing a large scale application development, in particular the implementation of mobile application along with developing from scratch a enterprise style database to keep hold of all of the information that is needed for the application.\\
\\
I as developer have also learnt that problems arise in developing any application and the way you learn to deal with the issues and get past them is also an integral part of the development of any project, learning from the mistakes will ultimately make a developer a better developer. From this project I have taken away that I need to personally follow process better and not get so flustered by time constraints and try and develop the application to be the best it can be rather than getting the application to a presentable level for the given deadlines where it looks okay on the surface but under the covers it needs a lot of work to get it to a good standard where it can be released others would argue that what is there is up to the standard of most applications but personally I am not happy with the standard the code is at at the moment and will need more work.\\
\\
But I must end on a positive note and this project even though very hard at some points has been very rewarding and very interesting to work on and I would like to take the points learnt throughout the project and use them to further my self within the software development industry, and hopefully create a great product that people will want to use, at the end of the day what is a application without its users?