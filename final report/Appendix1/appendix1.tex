\chapter{Third-Party Code and Libraries}

\lstjava

\section{Android application}


\subsection{Libraries}

\subsubsection*{Google Maps API}

\subsubsection*{Android Asynchronous Http Client}


\subsection{Code Snippets}

\subsubsection*{GPS Code}

\begin{lstlisting}

package uk.co.tomrosier.xetk.losesono.prototype.prototype.utils;

import android.app.AlertDialog;
import android.app.Service;
import android.content.Context;
import android.content.DialogInterface;
import android.content.Intent;
import android.location.Location;
import android.location.LocationListener;
import android.location.LocationManager;
import android.os.Bundle;
import android.os.IBinder;
import android.provider.Settings;
import android.util.Log;

/**
 * TODO: power saving needs to be implemented.
 * This handles interaction with the Android GPS and location services.
 */

// Taken from http://stackoverflow.com/a/15757944

public class GPSTracker extends Service implements LocationListener {

    private final Context mContext;

    // Flag for GPS status
    boolean isGPSEnabled = false;

    // Flag for network status
    boolean isNetworkEnabled = false;

    // Flag for GPS status
    boolean canGetLocation = false;

    Location location; // Location
    double latitude; // Latitude
    double longitude; // Longitude

    // The minimum distance to change Updates in meters
    private static final long MIN_DISTANCE_CHANGE_FOR_UPDATES = 10; // 10 meters

    // The minimum time between updates in milliseconds
    private static final long MIN_TIME_BW_UPDATES = 1000 * 60 * 1; // 1 minute

    // Declaring a Location Manager
    protected LocationManager locationManager;

    public GPSTracker(Context context) {
        this.mContext = context;
        getLocation();
    }

    public Location getLocation() {
        try {

            locationManager = (LocationManager) mContext
                    .getSystemService(LOCATION_SERVICE);

            // Getting GPS status
            isGPSEnabled = locationManager
                    .isProviderEnabled(LocationManager.GPS_PROVIDER);

            // Getting network status

            if (!isGPSEnabled && !isNetworkEnabled) {
                // No network provider is enabled
            } else {
                this.canGetLocation = true;
                if (isNetworkEnabled) {

                    locationManager.requestLocationUpdates(
                            LocationManager.NETWORK_PROVIDER,
                            MIN_TIME_BW_UPDATES,
                            MIN_DISTANCE_CHANGE_FOR_UPDATES, this);

                    Log.d("Network", "Network");

                    if (locationManager != null) {
                        location = locationManager
                                .getLastKnownLocation(LocationManager.NETWORK_PROVIDER);

                        if (location != null) {
                            latitude = location.getLatitude();
                            longitude = location.getLongitude();
                        }
                    }
                }

                // If GPS enabled, get latitude/longitude using GPS Services
                if (isGPSEnabled) {

                    if (location == null) {

                        locationManager.requestLocationUpdates(
                                LocationManager.GPS_PROVIDER,
                                MIN_TIME_BW_UPDATES,
                                MIN_DISTANCE_CHANGE_FOR_UPDATES, this);

                        Log.d("GPS Enabled", "GPS Enabled");

                        if (locationManager != null) {
                            location = locationManager
                                    .getLastKnownLocation(LocationManager.GPS_PROVIDER);

                            if (location != null) {
                                latitude = location.getLatitude();
                                longitude = location.getLongitude();
                            }
                        }
                    }
                }
            }
        }
        catch (Exception e) {
            e.printStackTrace();
        }

        return location;
    }


    /**
     * Stop using GPS listener
     * Calling this function will stop using GPS in your app.
     * */
    public void stopUsingGPS(){
        if(locationManager != null){
            locationManager.removeUpdates(GPSTracker.this);
        }
    }


    /**
     * Function to get latitude
     * */
    public double getLatitude(){
        if(location != null){
            latitude = location.getLatitude();
        }

        // return latitude
        return latitude;
    }


    /**
     * Function to get longitude
     * */
    public double getLongitude(){
        if(location != null){
            longitude = location.getLongitude();
        }

        // return longitude
        return longitude;
    }

    /**
     * Function to check GPS/Wi-Fi enabled
     * @return boolean
     * */
    public boolean canGetLocation() {
        return this.canGetLocation;
    }


    /**
     * Function to show settings alert dialog.
     * On pressing the Settings button it will launch Settings Options.
     * */
    public void showSettingsAlert(){
        AlertDialog.Builder alertDialog = new AlertDialog.Builder(mContext);

        // Setting Dialog Title
        alertDialog.setTitle("GPS is settings");

        // Setting Dialog Message
        alertDialog.setMessage("GPS is not enabled. Do you want to go to settings menu?");

        // On pressing the Settings button.
        alertDialog.setPositiveButton("Settings", new DialogInterface.OnClickListener() {
            public void onClick(DialogInterface dialog,int which) {
                Intent intent = new Intent(Settings.ACTION_LOCATION_SOURCE_SETTINGS);
                mContext.startActivity(intent);
            }
        });

        // On pressing the cancel button
        alertDialog.setNegativeButton("Cancel", new DialogInterface.OnClickListener() {
            public void onClick(DialogInterface dialog, int which) {
                dialog.cancel();
            }
        });

        // Showing Alert Message
        alertDialog.show();
    }


    @Override
    public void onLocationChanged(Location location) {
    }


    @Override
    public void onProviderDisabled(String provider) {
    }


    @Override
    public void onProviderEnabled(String provider) {
    }


    @Override
    public void onStatusChanged(String provider, int status, Bundle extras) {
    }


    @Override
    public IBinder onBind(Intent arg0) {
        return null;
    }
}

\end{lstlisting}


\subsubsection*{GPS Distance Calculation Code}

\begin{lstlisting}

// This calculates the distance between two gps coordinates.
// Based off http://www.movable-type.co.uk/scripts/latlong.html.
private static double calculateDistance(Double lat1, Double lon1, Double lat2, Double lon2) {

    // The radius of the earth.
    long radius = 6371000;

    // Seriously do not have a idea whats going on here, Copied it from the movable-type source and just checked the variables had the same values.

    double la1 = Math.toRadians(lat1);
    double la2 = Math.toRadians(lat2);

    double lo1 = Math.toRadians(lon1);
    double lo2 = Math.toRadians(lon2);

    double lat = (la2 - la1);
    double lon = (lo2 - lo1);

    double a = Math.sin(lat / 2) * Math.sin(lat / 2) +
               Math.cos(la1)     * Math.cos(la2) *
               Math.sin(lon / 2) * Math.sin(lon / 2);

    double c = 2 * Math.atan2(Math.sqrt(a), Math.sqrt(1-a));

    double distant = radius * c;

    return distant;
}
 
\end{lstlisting}


\subsubsection*{Google Maps API Code}

\begin{lstlisting}

    // This sets up map ready.
    private void setUpMapIfNeeded() {

        /* Side note, all Google's API documentation is out of date for this... */

        // Try to obtain the map from the SupportMapFragment.
        MapFragment mapFragment = (MapFragment) getFragmentManager().findFragmentById(R.id.messageMapFragment);

        // Setup the map asynchronously,
        mapFragment.getMapAsync(
            new OnMapReadyCallback() {

                // When the map has loaded then we do stuff thats important.
                @Override
                public void onMapReady(GoogleMap googleMap) {
                    // Setup the map for the needs we need.
                    setUpMap(googleMap);
                }
            }
        );
    }

    // Setup the map with the things important to us.
    private void setUpMap(GoogleMap mMap) {

        // Get the current location so we can tag it on the map.
        HandleGPS gps = new HandleGPS(this);

        // Set the current lat and long, so we can access them again later.
        lat = gps.getLatitude();
        lon = gps.getLongitude();

        // Check we have a valid GPS fix before we populate the maps.
        boolean validGPSFix = gps.isValidGPS();

        if (validGPSFix) {

            // Gen a new lat and long ready to make the marker.
            LatLng location = new LatLng(lat, lon);

            // The specific options for the tag.
            MarkerOptions markerOpts = new MarkerOptions();
            markerOpts.position(location);
            markerOpts.title("Add a tag to this location.");
            markerOpts.visible(true);

            // Get a marker from the options.
            Marker marker = mMap.addMarker(markerOpts);
            marker.showInfoWindow(); // Show the information handle.

            // Setup the view point for the map.
            mMap.moveCamera(CameraUpdateFactory.newLatLngZoom(location, 16));

            // Disable the things we don't need.
            mMap.getUiSettings().setScrollGesturesEnabled(false);
            mMap.getUiSettings().setZoomGesturesEnabled(false);
            mMap.getUiSettings().setMapToolbarEnabled(false);
            mMap.getUiSettings().setCompassEnabled(false);
            mMap.getUiSettings().setRotateGesturesEnabled(false);
            mMap.getUiSettings().setTiltGesturesEnabled(false);
        }
    }
 
\end{lstlisting}


\subsubsection*{Notification Code}

GPSService.java
\begin{lstlisting}
 
    // This creates a notification that can be viewable in the Android notification draw.
    private void createNotification(Message msg, User user) {

        System.out.println("Creating notification");

        // Get the message ID from the message to make it easy to access.
        int msgID = msg.getMessageID();

        // Create some pending intents needed for the notification, one thats for if the notification is dismissed and one for if we want to view the notification.
        PendingIntent dismissIntent = NotificationActivity.getDismissIntent(msgID, context);
        PendingIntent realIntent    = NotificationActivity.getRealIntent(msgID,    context);

        // Build the string that is displayed in the notification preview.
        String msgStr = msg.getContent() + " - " + user.getFirstName() + " " + user.getLastName();

        // Build the notification with the details we need.
        NotificationCompat.Builder builder = new NotificationCompat.Builder(context);

        builder .setDefaults(Notification.DEFAULT_ALL) // also requires VIBRATE permission
                .setSmallIcon(R.drawable.icon) // Required!
                .setContentTitle("Tag nearby")
                .setContentText(msgStr)
                .setAutoCancel(true)
                .setOnlyAlertOnce(true)
                .addAction(0, "View Tag", realIntent)
                .addAction(0, "Dismiss Tag", dismissIntent);

        // Builds the notification and issues it.
        notifyMgr.notify(msgID, builder.build());
    }
\end{lstlisting}

NotificationActivity.java

\begin{lstlisting}

package uk.co.tomrosier.xetk.losesono.prototype.prototype.activities;

import android.app.Activity;
import android.app.NotificationManager;
import android.app.PendingIntent;
import android.app.TaskStackBuilder;
import android.content.Context;
import android.content.Intent;
import android.os.Bundle;
import android.widget.Toast;

/**
 * This handles the dealing with notifications, as we decide if we want to either clear the notification or we go to the tag on the map.
 */
public class NotificationActivity extends Activity {

    // The id of the notification.
    public static final String NOTIFICATION_ID = "NOTIFICATION_ID";

    @Override
    protected void onCreate(Bundle savedInstanceState) {
        super.onCreate(savedInstanceState);

        // This gets the list of notifications that have already been created.
        NotificationManager manager = (NotificationManager) getSystemService(NOTIFICATION_SERVICE);

        // Get the id of the message that we are working with that is the notification.
        int msgID = getIntent().getIntExtra(NOTIFICATION_ID, -1);

        // Clear the notification from the list that shows the notifications.
        manager.cancel(msgID);

        // Show that we are clearing all of the notificiations.
        Toast.makeText(getApplicationContext(), "Clearing Notification: " + msgID, Toast.LENGTH_LONG).show();

        // Close this activity when we are done.
        finish(); // since finish() is called in onCreate(), onDestroy() will be called immediately
    }

    // This clears the notification and marks it as already read.
    public static PendingIntent getDismissIntent(int notificationId, Context context) {

        // Get the activity ready which is this activity.
        Intent intent = new Intent(context, NotificationActivity.class);

        //intent.setFlags(Intent.FLAG_ACTIVITY_NEW_TASK | Intent.FLAG_ACTIVITY_CLEAR_TASK);

        // Put the id's of the notification we want to cancel.
        intent.putExtra(NOTIFICATION_ID, notificationId);
        intent.setAction("Notification: " + notificationId);

        // Create the pending intent ready for us to clear it.
        PendingIntent dismissIntent = PendingIntent.getActivity(context, 0, intent, PendingIntent.FLAG_CANCEL_CURRENT);

        // Return the pending intent.
        return dismissIntent;
    }

    // This loads the activity to display the notification.
    public static PendingIntent getRealIntent(int messageID, Context context) {

        // Create a intent to the activity that displays the messages.
        Intent resultIntent = new Intent(context, ViewMessage.class);

        // Pass the message_id to the activity to view the message that has been left.
        resultIntent.putExtra("MsgObj", messageID);
        resultIntent.setAction("Notification: " + messageID);

        // Create a stack that holds the other activityies that need to be in the stack.
        TaskStackBuilder stackBuilder = TaskStackBuilder.create(context);

        // Put a activity to fall back once we have finished with the message.
        stackBuilder.addParentStack(NavigationActivity.class);

        // Add the pending intent.
        stackBuilder.addNextIntent(resultIntent);

        // return the pending intent.
        return stackBuilder.getPendingIntent(0, PendingIntent.FLAG_ONE_SHOT);
    }
}

\end{lstlisting}



\subsubsection*{Background Services Code}

DiscoverService.java

\begin{lstlisting}

package uk.co.tomrosier.xetk.losesono.prototype.prototype.services;

import android.app.AlarmManager;
import android.app.PendingIntent;
import android.app.Service;
import android.content.Intent;
import android.os.IBinder;

/**
 * This is the service that runs every so often to check if the user is close by to a specific message.
 */
public class DiscoverService extends Service {

    // Set the delay in secounds between checking again for new notifications.
    private final static int delay = 30;

    @Override
    public IBinder onBind(Intent intent) {
        return null;
    }

    // Load up the service when the services are deployed.
    @Override
    public int onStartCommand(Intent intent, int flags, int startId) {

        System.out.println("Service has been run");

        // Get the current GPS location and check for changes.
        GPSService gPSS = new GPSService(getApplicationContext());

        gPSS.checkForChange();

        stopSelf();
        return START_NOT_STICKY;
    }

    @Override
    public void onCreate() {
        super.onCreate();

        // Needed to debug services.
        // android.os.Debug.waitForDebugger();

    }

    @Override
    public void onDestroy() {
        // Setup alarm to run services again in x amount of time.
        AlarmManager alarm = (AlarmManager)getSystemService(ALARM_SERVICE);

        Intent resumeService = new Intent(this, DiscoverService.class);

        alarm.set(
                alarm.RTC_WAKEUP,
                System.currentTimeMillis() + (delay * 1000),
                PendingIntent.getService(this, 0, resumeService, 0)
        );
    }
}

\end{lstlisting}

BootReceiver.java

\begin{lstlisting}

package uk.co.tomrosier.xetk.losesono.prototype.prototype.services;

import android.content.BroadcastReceiver;
import android.content.Context;
import android.content.Intent;

/**
 * This class detects when the application is started.
 */
public class BootReceiver extends BroadcastReceiver {

    @Override
    public void onReceive(Context context, Intent intent) {

        // When the application is installed or the phone has finished booting then run this code.
        if (intent.getAction().equals(Intent.ACTION_BOOT_COMPLETED) ||
                intent.getAction().equals(Intent.ACTION_MY_PACKAGE_REPLACED)) {

            // Start the service to detect when a message is near.
            Intent service = new Intent(context, DiscoverService.class);
            context.startService(service);
        }
    }
}

\end{lstlisting}

\subsubsection*{Listview Code}

MessageFriendsActivity.java

\begin{lstlisting}
package uk.co.tomrosier.xetk.losesono.prototype.prototype.activities;

import android.content.Intent;
import android.os.Bundle;
import android.support.v7.app.ActionBarActivity;
import android.view.Menu;
import android.view.MenuItem;
import android.view.View;
import android.widget.Button;
import android.widget.ListView;

import java.util.ArrayList;

import uk.co.tomrosier.xetk.losesono.prototype.prototype.ArrayAdapters.FriendArrayAdapter;
import uk.co.tomrosier.xetk.losesono.prototype.prototype.R;
import uk.co.tomrosier.xetk.losesono.prototype.prototype.RestClient.FriendRestClient;
import uk.co.tomrosier.xetk.losesono.prototype.prototype.RestClient.MessageRestClient;
import uk.co.tomrosier.xetk.losesono.prototype.prototype.entities.Friend;
import uk.co.tomrosier.xetk.losesono.prototype.prototype.utils.AjaxCompleteHandler;

/**
 * This is the activity for adding the visibility to friends for a given message.
 */

public class MessageFriendsActivity extends ActionBarActivity {

    // List of the items shown on the screen.
    private ArrayList<Friend> listItems = new ArrayList<Friend>();

    /// This is the adapter that shows the listview on the screen.
    private FriendArrayAdapter adapter;

    @Override
    protected void onCreate(Bundle savedInstanceState) {
        super.onCreate(savedInstanceState);
        setContentView(R.layout.activity_message_friends);

        // Get the list view from the layout so we can add items to it.
        ListView lw = (ListView) findViewById(R.id.ListViewFriends);

        // Setup the display adapter to add the items to.
        adapter = new FriendArrayAdapter(this, listItems);

        // Set the view point for the listview.
        lw.setAdapter(adapter);

        // Get the bundle from the intent.
        Bundle extras = getIntent().getExtras();

        // Check the extras see if we have any.
        if (extras != null) {

            // Get the message_id that has been passed to the activity.
            final int messageID = extras.getInt("Message_ID");

            // Get the restclient for dealing with friends.
            FriendRestClient fRC = new FriendRestClient(this);

            // Get the list of friends from the server.
            fRC.getFriends(
                new AjaxCompleteHandler() {
                    @Override
                    public void handleAction(Object someData) {
                        // What we do with each friends in the list.
                        Friend user = (Friend)someData;

                        // Add each friend to the list.
                        listItems.add(user);
                        // Show the data has been changed once we added.
                        adapter.notifyDataSetChanged();
                    }
                }
            );

            // Get the button for submitting the data.
            Button BtnSelectFriends = (Button) findViewById(R.id.BtnSelectFriends);

            // Process the selected friends on the button press.
            BtnSelectFriends.setOnClickListener(
                    new View.OnClickListener() {

                        @Override
                        public void onClick(View arg0) {
                            // For each of the friends, in the list view.
                            for (int i = 0; i < adapter.getCount(); i++) {
                                // Get the friend object from the adapter view.
                                Friend fi = adapter.getItem(i);

                                if (fi.isChecked()) {
                                    // Get the message rest client ready to append the friend to the message.
                                    MessageRestClient mRC = new MessageRestClient(getApplicationContext());

                                    // Add the user to the message.
                                    mRC.addUser(
                                            new AjaxCompleteHandler() {
                                                @Override
                                                public void handleAction(Object someData) {
                                                    int groupID = (int) someData;
                                                    System.out.println("GroupID: " + groupID);
                                                }
                                            },
                                            messageID,
                                            fi.getLinkID()
                                    );
                                }
                            }
                            // Close the activity.
                            finish();
                            // Go back to the home activity.
                            Intent intent = new Intent(getApplicationContext(), NavigationActivity.class);
                            startActivity(intent);
                        }

                    }
            );
        }
    }

    @Override
    public boolean onCreateOptionsMenu(Menu menu) {
        // Inflate the menu; this adds items to the action bar if it is present.
        getMenuInflater().inflate(R.menu.menu_message_friends, menu);
        return super.onCreateOptionsMenu(menu);
    }

    @Override
    public boolean onOptionsItemSelected(MenuItem item) {
        // Handle action bar item clicks here. The action bar will
        // automatically handle clicks on the Home/Up button, so long
        // as you specify a parent activity in AndroidManifest.xml.
        int id = item.getItemId();

        //noinspection SimplifiableIfStatement
        if (id == R.id.add_friend) {

            // Show the dialog for adding a new friends.
            Intent myIntent = new Intent(MessageFriendsActivity.this, AddFriendActivity.class);
            MessageFriendsActivity.this.startActivity(myIntent);

            return true;
        }

        return super.onOptionsItemSelected(item);
    }
}
\end{lstlisting}

FriendArrayAdapter.java
\begin{lstlisting}
package uk.co.tomrosier.xetk.losesono.prototype.prototype.ArrayAdapters;

import android.content.Context;
import android.view.LayoutInflater;
import android.view.View;
import android.view.ViewGroup;
import android.widget.ArrayAdapter;
import android.widget.CheckBox;
import android.widget.TextView;

import java.util.ArrayList;

import uk.co.tomrosier.xetk.losesono.prototype.prototype.R;
import uk.co.tomrosier.xetk.losesono.prototype.prototype.entities.Friend;

/**
 * This is for the listview on the adding friends to a message page.
 */
public class FriendArrayAdapter extends ArrayAdapter<Friend> {

    // Keep track of the friends that our shown on the page.
    private ArrayList<Friend> friendList = new ArrayList<Friend>();

    // When we instantiate the listview we want to get everything ready, for action.
    public FriendArrayAdapter(Context context, ArrayList<Friend> friends) {
        super(context, 0, friends);
        this.friendList = new ArrayList<Friend>();
        this.friendList.addAll(friendList);
    }

    @Override
    public View getView(int position, View convertView, ViewGroup parent) {
        // Get the data item for this position
        final Friend user = getItem(position);
        // Check if an existing view is being reused, otherwise inflate the view

        if (convertView == null) {
            convertView = LayoutInflater.from(getContext()).inflate(R.layout.friend_list_item, parent, false);
        }

        // Lookup view for data population
        TextView friendName = (TextView) convertView.findViewById(R.id.friendName);
        CheckBox friendCB   = (CheckBox) convertView.findViewById(R.id.friendCB);

        // String that is displayed.
        String name = "(" + user.getUser().getUserNmae() + ") " + user.getUser().getFirstName() + " " + user.getUser().getLastName();


        // Populate the data into the template view using the data object
        friendName.setText(name);
        friendCB.setChecked(false);

        // Register the onClick. To check the box.
        friendCB.setOnClickListener(
            new View.OnClickListener() {
                @Override
                public void onClick(View v) {
                    user.setChecked(!user.isChecked());
                }
            }
        );

        // Return the completed view to render on screen
        return convertView;
    }

    // This gets the id for a given item.
    @Override
    public long getItemId(int position) {
        Friend user = getItem(position);
        return user.getUser().getUserID();
    }

    @Override
    public boolean hasStableIds() {
        return true;
    }
}
\end{lstlisting}

\subsubsection*{Miscellaneous}



\section{Server side}


\subsection{Libraries}

\subsubsection*{HAPI.js}

\subsubsection*{Sequelize}


\subsection{Code Snippets}

\subsubsection*{JSON 2 SQL}

\lstsql
\begin{lstlisting}

/*
 * This is a method taken from Stack Overflow, it converts JSON objects directly into records within tables and inserts them into the table.
 * Taken from http://stackoverflow.com/a/28573007 
 * All credit goes to user "beldaz".
 */
create or replace function jsoninsert(relname text, reljson text) returns record 
as
  $body$declare
  ret record;
  inputstring text;
begin

  select string_agg(quote_ident(key),',') 
  into   inputstring
  from   json_object_keys(reljson::json) as x (key);

  execute 'insert into '|| quote_ident(relname) 
    || '(' || inputstring || ') select ' ||  inputstring 
    || ' from json_populate_record( null::' || quote_ident(relname) || ' , json_in($1)) returning *'
    into ret using reljson::cstring;

  return ret;
end;
$body$
language plpgsql volatile;

\end{lstlisting}


\subsubsection*{RouteLoader.js}

\lstjavascript
\begin{lstlisting}
// Resources used to create this dynamic loader.
// https://stackoverflow.com/questions/10914751/loading-node-js-modules-dynamically-based-on-route
// https://gist.github.com/kethinov/6658166
// https://github.com/XeTK/JSBot/blob/master/loader.js

// Import some stuff todo with file management and building paths.
var fs          = require('fs');
var path_module = require('path');

// Store for all the routes we load.
var route_holder = {};

// Build the path to the routes directory, making it not OS dependent.
var path = path_module.join(__dirname, 'routes');

// Post the path to the console that everyone knows where it is.
console.log('Routes path: ' + path.yellow);

// Function to read the whole directory tree from a specific part.
var walkSync = function(dir, filelist) {

    // Read all the files in the current directory we are working on.
    var files = fs.readdirSync(dir);

    // Add a new array onto the file list ready to add the next bunch of files.
    filelist = filelist || [];

    // For all the items we find in the directory.
    files.forEach(
        function(file) {

            // Build the whole path for a given file.
            var iPath = path_module.join(dir, file);

            // Check if the item is a directory.
            if (fs.statSync(iPath).isDirectory()) {
                // We start this all over again and get all the files in the directory.
                filelist = walkSync(iPath + '/', filelist);
            } else {
                // Push the files onto the file list.
                filelist.push(iPath);
            }
        }
    );

    // Return the file list that we just built.
    return filelist;
}; 

// Start reading the files in the routes directory.
var files = walkSync(path + '/');

// For each file we need to load them dynamically.
files.forEach(
    function(file) {
        
        // Regex we need to check the files are valid for loading.
        var regex = /\.js$/g;

        // If the file is valid and we can load it then we do.
        if (regex.test(file)) {
            // Give some dialog to the user so they can see what file has been loaded.
            console.log(('Loading route: ' + file).green);
            // Load the file using require into the application.
            require(file)(route_holder);
        } else {
            // If its not valid we discard it.
            console.log(('Not route: ' + file).red);
        }
    }
);

// Return the list of plugin's that have been loaded.
exports.route_holder = route_holder;
\end{lstlisting}