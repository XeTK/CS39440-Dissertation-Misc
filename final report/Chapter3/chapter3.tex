\chapter{Implementation}

This chapter covers in details the implementation of the features covered within the design chapter and will list in detail the general implementation of a feature including there issues, it will also detail the environments and tools used to create the application and the supporting services behind the application.

\section{Tools used}

In this section the tools used for building the application and backend services will be detailed and there role within the development of the project as a whole.

\subsection*{Sublime Text}

Sublime Text is one of the main tools that was used throughout the development of the project, it is highly regarded as one of the best text editors in the public space due to its flexibility and community support to help it fit the users needs.\\
\\
It has been a crucial tool for developing the middle tier application as all of the code was written in sublime text, this report that you are reading now was exclusively written in Sublime Text and the vast majority of the SQL was also writing within the Sublime Text editor. In image of the text editor can be found in figure \ref{fig:sublime_text_image}.

\begin{figure}[H]
    \centering
    \includegraphics[width=\textwidth]{tools/sublime}
    \caption{Sublime Text 3.0}
    \label{fig:sublime_text_image}
\end{figure} 

\subsection*{Android Studio}

Android Studio has played the most crucial role within the development of the project it has been the integrated development environment of choice for Android development giving all the required features and tools needed to create a fully fledged Android application with the potential of being released into a production environment, it contains all of the testing tools required to do good Android development with a unit testing suite and various UI testing suites.\\
\\
It has close links to the Android SDK so it enables users to use the various Android SDK tools within the Android Studio GUI some of the advanced tools within the Android SDK are covered later in this section. Android Studio offers powerful code completion tools and refactoring tools to speed up the development of the application it also has the ability to error check the code and provide solutions through the development of the project with powerful auto complete tools to build up the stub functions to complete the implementation of imported libraries. An example of the layout and design of Android studio can be found in figure \ref{fig:android_studio_image}.

\begin{figure}[H]
    \centering
    \includegraphics[width=\textwidth]{tools/androidstudio}
    \caption{Android Studio 1.1.0}
    \label{fig:android_studio_image}
\end{figure} 

\subsection*{PGAdmin3}

PGAdmin3 is a tool for remotely accessing Postgres database and allowing management through a GUI based interface, it allows easy access to viewing data held within the database along with the ability to execute SQL commands directly on the database. The tool was mostly used as a verifier to check that SQL executed within the terminal had run correctly and had created the tables and constraints that were needed for the application to work correctly. Small note this crashes a lot on a Apple Mac, An example of the execution of PGAdmin3 can be found in figure \ref{fig:pg_admin_image}.

\begin{figure}[H]
    \centering
    \includegraphics[width=\textwidth]{tools/pgadmin}
    \caption{PGAdmin3 1.20.0}
    \label{fig:pg_admin_image}
\end{figure} 

\subsection*{PGSQL command line}

The PostgreSQL command line tool was used in conjunction with PGAdmin3 to do all the SQL related work needed for the project, the PostgeSQL command line allows direct command execution within the SQL engine so is perfect for debugging issues within the database along with creating the SQL statements to extract the data needed to get the functional requirements of the application working correctly.\\
\\
PostgreSQL its self is fairly difficult to install correctly and there were many issues with getting it to install correctly and work as intended, the command line package manager for Apple Mac Homebrew PostgeSQL's package is completely broken and caused massive headaches at the start of the project theses issues were resolved by installing the official DMG package provided by Postgres.app \cite{jemt:postgresapp:2015:online}.

\begin{figure}[H]
    \centering
    \includegraphics[width=\textwidth]{tools/pgsqlcommandline}
    \caption{PG SQL Command line 9.3.6}
    \label{fig:pg_sql_image}
\end{figure} 

\subsection*{Postman}

Postman was an essential tool for debugging the RESTful interfaces provided by the applications, it enables the emulation of HTTP POST and GET requests enabling the ability to attach the parameters needed when completing a POST request and making sure the interface is working correctly without the need to implementing it fully into the application. It was often used to check if the RESTful interface was working correctly before implementing the corresponding code within the Android side of the application, Postman its self has a few issues, if a request would fail for what ever reason the application would get stuck in the sending request stage and would require closing the tab and re opening. An image of Postman at work can be found in figure \ref{fig:postman_image}.

\begin{figure}[H]
    \centering
    \includegraphics[width=\textwidth]{tools/postman}
    \caption{Postman 2.0.19}
    \label{fig:postman_image}
\end{figure} 

\subsection*{Web browsers}

A web browser is an essential tool in any software related tool in the modern world, It was primary used for researching the project as a whole but it was also used for debugging the backend API as it can carry out HTTP requests, accept cookies and deal with HTTP authentication. They were often used to complement Postman when it could not quite carry out the required action that was needed.\\
\\
The two main browsers used to test the application were Opera And Google Chrome, Opera being my main browser so Chrome was used as a backup when needed. Images of the two main browsers that were used can be found in figures \ref{fig:opera_image} and \ref{fig:chrome_image}.

\begin{figure}[H]
    \centering
    \includegraphics[width=\textwidth]{tools/opera}
    \caption{Opera 28.0}
    \label{fig:opera_image}
\end{figure} 

\begin{figure}[H]
    \centering
    \includegraphics[width=\textwidth]{tools/chrome}
    \caption{Chrome 43.0}
    \label{fig:chrome_image}
\end{figure} 

\subsection*{Android SDK Tools}

Throughout the development of the project there was a need to use the full Android Software Development Kit for testing and debugging the application in the given environment it will be run. The SDK provides all of the build tools needed to create and compile the packages needed to install the application on the device along with submit it to the Google Play Store.\\
\\
The first and possible most used tool throughout the development of the application is the Android Emulator, the tool emulates the Android platform on the development machine to provide an environment to test code without the need of a physical device this can be useful when the developer only has a limited selection of devices to test the application on as the emulator gives the ability to scale to different screen sizes and resolutions to help with debugging on different devices. In figure \ref{fig:android_emulator} shows the emulator at work running the application in a test mode.

\begin{figure}[H]
    \centering
    \includegraphics[width=0.5\textwidth]{tools/androidemulator}
    \caption{Android Emulator SDK 22}
    \label{fig:android_emulator}
\end{figure} 

\noindent
Another equally important tool that was used extensively during the development of the project was the command line application Android Development Bridge(ADB). ADB is used to interface with Android based devices and provides many features that allow interfacing and connecting to the device, it is crucial for debugging applications along with working with the Operating System directly. It has been extensively used through this project to repair damaged and broken phones to bring them back to a state where they can be used to help with development.

\begin{figure}[H]
    \centering
    \includegraphics[width=0.75\textwidth]{tools/adb}
    \caption{Android Development Bridge}
    \label{fig:adb_image}
\end{figure} 

\noindent
The SDK updater played a very minor in the project but it allows easy and quick way to update the SDK and build tools within the Android development suite to the latest versions to enable development for newer devices as they come to market.

\begin{figure}[H]
    \centering
    \includegraphics[width=0.75\textwidth]{tools/sdkupdator}
    \caption{Android SDK Updater}
    \label{fig:sdk_updator}
\end{figure} 

\subsection*{Git Hub}

Throughout the project the version control system GIT has been used to ensure that the source code for the project has been reliably backed up and provide the ability to check in versions of the code to return to a prior point if needed.\\
\\
The use of the service GitHub as a on-line repository to keep the code easily accessible and backed up throughout the project along using some of the extra functionality by using a 3rd party to store and look after the source code. In future it will enable people to contribute to the project and do there own tweaks and improvements to ensure that the project becomes a fully fledged and vibrant application. The standard GitHub page layout can be found in figure \ref{fig:git_hub_repos_image}.

\begin{figure}[H]
    \centering
    \includegraphics[width=\textwidth]{tools/github}
    \caption{Git Hub git repositories}
    \label{fig:git_hub_repos_image}
\end{figure} 

\noindent
GitHub provides a issue tracker so that users can leave issues and bug reports for the developer to fix or improve, it has been integral throughout the project to keep track of any issues that have been raised within the development of the application. They have been marked as fixed or left ready to be fixed in the future development of the application and gives a traceable history of any issues within the application.

\begin{figure}[H]
    \centering
    \includegraphics[width=\textwidth]{tools/githubissues}
    \caption{Github issue tracker}
    \label{fig:gh_issue_tracker_image}
\end{figure} 

\section{Android application}

\subsection{Development Hardware}

\subsection{Environment}

\subsection{Features}


\subsubsection*{Login}

\paragraph*{Implementation}

\paragraph*{Issues}


\subsubsection*{Registration}

\paragraph*{Implementation}

\paragraph*{Issues}


\subsubsection*{Adding Friends}

\paragraph*{Implementation}

\paragraph*{Issues}


\subsubsection*{GPS Location}

\paragraph*{Implementation}

\paragraph*{Issues}


\subsubsection*{Maps}

\paragraph*{Implementation}

\paragraph*{Issues}


\subsubsection*{Posting message}

\paragraph*{Implementation}

\paragraph*{Issues}


\subsubsection*{Retrieving messages}

\paragraph*{Implementation}

\paragraph*{Issues}


\subsubsection*{Retrieving notifications}

\paragraph*{Implementation}

\paragraph*{Issues}


\subsubsection*{Comments}

\paragraph*{Implementation}

\paragraph*{Issues}


\subsubsection*{Votes}

\paragraph*{Implementation}

\paragraph*{Issues}



\section{Server side application}


\subsection{Environment}

\subsubsection{Debian based Linux}

\subsubsection{Postgres Database}

\subsubsection{Node.js environment}

\subsubsection{Digital Ocean Droplet}

% Ram issues


\subsection{Middle tier application}

\subsubsection{Core functionality}

\subsubsection{Rest Interface}

\subsubsection{Database Connector}


\subsection{Database level}

\subsubsection{Tables}

\subsubsection{Functions}


\section{Review of implementation}