\documentclass[11pt,fleqn,twoside]{article}
\usepackage{makeidx}
\makeindex
\usepackage{palatino} %or {times} etc
\usepackage{plain} %bibliography style 
\usepackage{amsmath} %math fonts - just in case
\usepackage{amsfonts} %math fonts
\usepackage{amssymb} %math fonts
\usepackage{lastpage} %for footer page numbers
\usepackage{fancyhdr} %header and footer package
\usepackage{mmpv2} 
\usepackage{url}

% the following packages are used for citations - You only need to include one. 
%
% Use the cite package if you are using the numeric style (e.g. IEEEannot). 
% Use the natbib package if you are using the author-date style (e.g. authordate2annot). 
% Only use one of these and comment out the other one. 
\usepackage{cite}
%\usepackage{natbib}

\begin{document}

\name{Thomas Mark Rosier}
\userid{THR2}
\projecttitle{GPS Based Location Sensitive Social Tagging Application}
\projecttitlememoir{Location Sensitive Social Notifier} %same as the project title or abridged version for page header
\reporttitle{Outline Project Specification}
\version{0.1}
\docstatus{Draft}
\modulecode{CS39440}
\degreeschemecode{G600}
\degreeschemename{Software Engineering}
\supervisor{David Price} % e.g. Neil Taylor
\supervisorid{DAP}
\wordcount{}

%optional - comment out next line to use current date for the document
%\documentdate{10th February 2014} 
\mmp

\setcounter{tocdepth}{3} %set required number of level in table of contents


%==============================================================================
\section{Project description}
%==============================================================================

My Major project is to create a social messaging platform that is based on the location based posts, the main idea is to be able to leave a message on a specific physical location and when one of your followers on the application walks over this specific location then a notification is thrown up on the recipients phone showing the message that has been left on this location. \\
\\
Some of the key applications of this application could also be to be a public notifier of information people in specific areas, So for example an event like a run there could be a marker at various points throughout the event where it notifiers the user the percentage they are through the run. \\
\\
A more mundane usage of the application could be to notify various members of a household to remind one of the other members to pick something up from the local shop when they reach the entrance of their work place. \\
\\
The application main talent should be that it is easy to post a new message without to much thought and most of the work should be hidden behind the scenes to make it as simple for the user to post messages, the user should be able to leave feedback on the tags that have been left (something a bit like a Facebook comments) with the ability to up vote and down vote various posts to give more social feedback on members posts.\\
\\
A considerable amount of work for the project will be determining if the post user is close to a post in the local area. And triggering notification to the user to give them the relevant information that they want to receive in the most efficient and battery friendly means possible, the problem with GPS is that it is incredibly power hungry and will drain the users battery very quickly if there is no effort to preserve it.\\
\\
I will develop the application in the native Android SDK, for efficiency and greater support my decision was aided by the fact I already own an Android based phone. And after talking to a few application developers I decided it is not wise to use phone gap as a lot of the functionality that I would require I would have to write my own custom libraries for to do the functionality that I was requiring.\\
\\




%==============================================================================
\section{Proposed tasks}
%==============================================================================


%==============================================================================
\section{Project deliverables}
%==============================================================================

%
% Start to comment out / remove the following lines. They are only provided for instruction for this example template.  You don't need the following section title, because it will be added as part of the bibliography section. 
%
%==============================================================================
\section*{Your Bibliography - REMOVE this title and text for final version}
%==============================================================================
%
You need to include an annotated bibliography. This should list all relevant web pages, books, journals etc. that you have consulted in researching your project. Each reference should include an annotation. 

The purpose of the section is to understand what sources you are looking at.  A correctly formatted list of items and annotations is sufficient. You might go further and make use of bibliographic tools, e.g. BibTeX in a LaTeX document, could be used to provide citations, for example \cite{NumericalRecipes} \cite{MarksPaper} \cite[99-101]{FailBlog} \cite{kittenpic_ref}.  The bibliographic tools are not a requirement, but you are welcome to use them.   

You can remove the above {\em Your Bibliography} section heading because it will be added in by the renewcommand which is part of the bibliography. The correct annotated bibliography information is provided below. 
%
% End of comment out / remove the lines. They are only provided for instruction for this example template. 
%


\nocite{*} % include everything from the bibliography, irrespective of whether it has been referenced.

% the following line is included so that the bibliography is also shown in the table of contents. There is the possibility that this is added to the previous page for the bibliography. To address this, a newline is added so that it appears on the first page for the bibliography. 
\newpage
\addcontentsline{toc}{section}{Initial Annotated Bibliography} 

%
% example of including an annotated bibliography. The current style is an author date one. If you want to change, comment out the line and uncomment the subsequent line. You should also modify the packages included at the top (see the notes earlier in the file) and then trash your aux files and re-run. 
%\bibliographystyle{authordate2annot}
\bibliographystyle{IEEEannot}
\renewcommand{\refname}{Annotated Bibliography}  % if you put text into the final {} on this line, you will get an extra title, e.g. References. This isn't necessary for the outline project specification. 
\bibliography{mmp} % References file

\end{document}
